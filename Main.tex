\documentclass[12pt, a4paper, oneside]{book}
\usepackage[utf8]{inputenc}
\usepackage{verbatim}
\usepackage{forloop}

%% --Figure settings--
\ifx\pdfoutput\undefined
%Use old Latex if PDFLatex does not work
\usepackage[dvips]{graphicx}% To get graphics working
\DeclareGraphicsExtensions{.eps} % Encapsulated PostScript
\else
%Use PDFLatex
\usepackage[pdftex]{graphicx}% To get graphics working
\DeclareGraphicsExtensions{.pdf,.jpg,.png,.bmp}
\pdfcompresslevel=9
\fi
\usepackage{subcaption}
\graphicspath{{Figures/}} %Path of figures
% -----------------------------------------
%%
\usepackage{minitoc}
\usepackage[table,xcdraw]{xcolor}
\usepackage{longtable}
\usepackage{makecell}
\usepackage{rotating}
\usepackage{tabulary}
\usepackage[titletoc]{appendix}
\usepackage{tikz}

%% --Font style setting--
\usepackage{indentfirst}
\usepackage{mathspec}
\setmainfont{Times New Roman}
\setmathsfont{Times New Roman}
\usepackage{xeCJK}
\usepackage{ifplatform} %Package for finding out which operating system is used.
\ifwindows
\setCJKmainfont{標楷體}
\else
\setCJKmainfont{BiauKai}
\fi
\XeTeXlinebreaklocale "zh"
\XeTeXlinebreakskip = 0pt plus 1pt
\defaultCJKfontfeatures{AutoFakeBold=6,AutoFakeSlant=.4}

% -----------------------------------------
%% Chinese Font bf setting
% -----------------------------------------
\def\xeCJKembold{0.4}

%% --hack into xeCJK, you don't need to understand it--
\def\saveCJKnode{\dimen255\lastkern}
\def\restoreCJKnode{\kern-\dimen255\kern\dimen255}

%% --save old definition of \CJKsymbol and \CJKpunctsymbol for CJK output--
\let\CJKoldsymbol\CJKsymbol
\let\CJKoldpunctsymbol\CJKpunctsymbol

%% --apply pdf literal fake bold--
\def\CJKfakeboldsymbol#1{%
	\special{pdf:literal direct 2 Tr \xeCJKembold\space w}%
	\CJKoldsymbol{#1}%
	\saveCJKnode
	\special{pdf:literal direct 0 Tr}%
	\restoreCJKnode}
\def\CJKfakeboldpunctsymbol#1{%
	\special{pdf:literal direct 2 Tr \xeCJKembold\space w}%
	\CJKoldpunctsymbol{#1}%
	\saveCJKnode
	\special{pdf:literal direct 0 Tr}%
	\restoreCJKnode}
\newcommand\CJKfakebold[1]{%
	\let\CJKsymbol\CJKfakeboldsymbol
	\let\CJKpunctsymbol\CJKfakeboldpunctsymbol
	#1%
	\let\CJKsymbol\CJKoldsymbol
	\let\CJKpunctsymbol\CJKoldpunctsymbol}
%%
%%
\usepackage{wallpaper}
\usepackage{setspace}
\usepackage{authblk}
\newsavebox{\largestimage}
\usepackage{amsmath,amssymb}
\usepackage{array}
\usepackage{tabularx}
\newcolumntype{L}[1]{>{\raggedright\let\newline\\\arraybackslash\hspace{0pt}}p{#1}}
\newcolumntype{C}[1]{>{\centering\let\newline\\\arraybackslash\hspace{0pt}}p{#1}}
\newcolumntype{R}[1]{>{\raggedleft\let\newline\\\arraybackslash\hspace{0pt}}b{#1}}
\allowdisplaybreaks
\usepackage[english]{babel}
\usepackage[square,numbers]{natbib}
%\usepackage{natbib}
\usepackage{multirow}
\usepackage[hidelinks]{hyperref}
\usepackage{breakcites}
\usepackage{geometry}
\usepackage[draft]{todonotes}   % notes showed
\usepackage{enumitem}
%% 
%% --Date setting for cover--
\usepackage{datetime} % For displaying month and year in the cover
\usepackage{fp} % To calculate the year of Minguo(民國)
\FPsub\Minguo{\the\year}{1911}
\FPeval{\Minguo}{round(Minguo,0)}

%% --Nomenclature of List of Symbols--
\usepackage{nomencl}
\makenomenclature
%% -----------------------------------------

\setcounter{secnumdepth}{4}
\setcounter{tocdepth}{1}
\usepackage{lipsum}

%%%%%%%%%%%%%%%%%%%%%%%%%%%%%%%%%%%%%%%%%%%%%%%%%%%%%%%%%%%%%%%%%%%%%
% Please create a new tex file and copy the content of this template%
%%%%%%%%%%%%%%%%%%%%%%%%%%%%%%%%%%%%%%%%%%%%%%%%%%%%%%%%%%%%%%%%%%%%%

\begin{document}
	\pagenumbering{gobble}
	\begin{titlepage}
		\newgeometry{a4paper,
		total={210mm,297mm},
		top=23mm,
		bottom=30mm,
		left=20mm,
		right=20mm,}
	{\Large
	\begin{center}
		\vspace*{0.1 cm}
		國 立 成 功 大 學\\
		\vspace{0.3 cm}
		機 械 工 程 學 系\\
		\vspace{0.3 cm}
		碩 士 論 文\\
		\vspace{2.5 cm}
		% Chinese title
		國立成功大學碩士用畢業論文LaTex模版\\
		\vspace{2.3 cm}
		% English title
		National Cheng Kung University (NCKU) Thesis Template in LaTex\\
		\vspace{2.3 cm}
	\end{center}
	\-\hspace{2cm}研\hspace{0.3 cm}究\hspace{0.3 cm}生:***\hspace{0.8 cm}Student :**** \\
	\-\hspace{2cm}指導教授:***\hspace{0.8 cm}Advisor :**** \\
	\begin{center}
		\vspace{1.5 cm}
		Department of Mechanical Engineering\\	
		National Cheng Kung University\\	
		Thesis for Master of Science\\	
		Tainan, Taiwan, R.O.C.\\	
		January \space \the\year\\
		\vspace{1.5 cm}
		中 華 民 國 \Minguo 年 \the\month 月 % Change \Minguo and \the\month to your defense year and month if it isn't satisfied
	\end{center}
}


	\ThisCenterWallPaper{.325}{NCKU_thesis_watermark.jpg}
	\end{titlepage}
	\newgeometry{
		a4paper,
		total={210mm,297mm},
		left=30mm,
		right=25mm,
		top=23mm,
		bottom=35mm,
	}
	\CenterWallPaper{.325}{NCKU_thesis_watermark.jpg}
	\begin{spacing}{1.5}
		\frontmatter
		\addcontentsline{toc}{chapter}{Chinese abstract}
		\chapter*{\centering Chinese abstract}
		\label{chap:chi_abstract}
		\CJKfakebold{研究背景:}

\CJKfakebold{研究目標:}

\CJKfakebold{研究方法:}

\CJKfakebold{研究結果:}

\CJKfakebold{研究結論:}

\CJKfakebold{關鍵字:}

		\newpage
		%% Abstract
		\addcontentsline{toc}{chapter}{Abstract}
		\chapter*{\centering Abstract}
		\label{chap:abstract}
		\textbf{Background:}

\textbf{Aim:}

\textbf{Method:}

\textbf{Results:}

\textbf{Conclusion:}

\textbf{Keywords:}
\textit{}
		\newpage
		\addcontentsline{toc}{chapter}{Acknowledgment}
		%% acknowledgment
		\chapter*{\centering Acknowledgment}
		\label{chap:acknowledgment}
		test \citep{kopka1995guide}.
\nomenclature{$a$}{Planck constant}

\noindent Author

\noindent 
		\newpage
		%% Table of contents
		\dominitoc 
		\begin{singlespace}
		\renewcommand{\contentsname}{Table of Contents}
		\cleardoublepage
		\addcontentsline{toc}{chapter}{Table of Contents}
		\tableofcontents
		%% List of tables
		\newpage
		\addcontentsline{toc}{chapter}{List of Tables}
		\listoftables
		%% List of figures
		\newpage
		\addcontentsline{toc}{chapter}{List of Figures}
		\listoffigures
		%% List of symbols
		\newpage
		\addcontentsline{toc}{chapter}{List of Symbols}
		\renewcommand{\nomname}{List of Symbols}
		\printnomenclature
		%% List of Acronyms
		% \chapter*{List of Acronyms}
		% \addcontentsline{toc}{chapter}{List of Acronyms}
		% \input{List_of_acronyms.tex}
		\end{singlespace}
		%% Main content
		\mainmatter

		\chapter{Introduction}
		\label{chap:Introduction}
		\section{Background}\label{sec:background}

\section{Motivation}\label{sec:motivation}

\section{Objective}\label{sec:objective}

\section{Thesis overview}\label{sec:overview}


		\chapter{Conclusion and future work}
		\label{chap:Conclusion}
		\section{Conclusion}

\section{Future works}
		
		\begin{singlespace}
		\clearpage
		%% References (bibliography)
		\addcontentsline{toc}{chapter}{References}
		\renewcommand{\bibname}{References}
		\bibliographystyle{apalike}
		\bibliography{library}% Name of .bib file
		\clearpage
		\end{singlespace}
		%% Appendix
		%\begin{appendices}
		%	\chapter{AppendixA}
		%	\label{app:A}
		%	\section{檔案結構 Directory tree}\label{sec:directory_tree}

\renewcommand\DTstyle{\rmfamily} %Setting font. Using times new roman in \dirtree
此 Directory tree 是使用 dirtree package 產生。
\dirtree{%
.1 /.
.2 Main.tex\DTcomment{主文件,所有設定皆在此}.
.2 Cover.tex\DTcomment{封面,請依需要修改}.
.2 Abstract\_en.tex\DTcomment{英文摘要}.
.2 Abstract\_ch.tex\DTcomment{中文摘要}.
.2 Acknowledgement.tex\DTcomment{致謝}.
.2 library.bib\DTcomment{參考文獻}.
.2 Figures\DTcomment{圖片資料夾}.
.3 NCKU\_thesis\_watermark.jpg\DTcomment{圖書館浮水印}.
}

\section{基本書寫 Basic writing}\label{sec:writing}
在每一個章節開始可以使用"\verb|\section{小節名稱}|"指令插入小節名稱,小節名稱會自動加到目錄中。
若目錄中想使用不同的小節名稱,則可以使用"\verb|\section[目錄顯示小節名稱]{內文顯示小節名稱}|"。


\section{如何插入圖片Insert figures}\label{sec:figure}

\section{標籤與引用 Label and references}\label{sec:Label_and_references}

\section{基本數學公式範例 Basic Example of Equation}\label{sec:Basic_Example_of_Equation}

		%\end{appendices}
		\backmatter
	\end{spacing}
\end{document}      