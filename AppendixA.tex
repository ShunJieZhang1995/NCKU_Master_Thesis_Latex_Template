\section{檔案結構 Directory tree}\label{sec:directory_tree}

\renewcommand\DTstyle{\rmfamily} %Setting font. Using times new roman in \dirtree
此 Directory tree 是使用 dirtree package 產生。
\dirtree{%
.1 /.
.2 Main.tex\DTcomment{主文件,所有設定皆在此}.
.2 Cover.tex\DTcomment{封面,請依需要修改}.
.2 Abstract\_en.tex\DTcomment{英文摘要}.
.2 Abstract\_ch.tex\DTcomment{中文摘要}.
.2 Acknowledgement.tex\DTcomment{致謝}.
.2 library.bib\DTcomment{參考文獻}.
.2 Figures\DTcomment{圖片資料夾}.
.3 NCKU\_thesis\_watermark.jpg\DTcomment{圖書館浮水印}.
}

\section{基本書寫 Basic writing}\label{sec:writing}
在每一個章節開始可以使用"\verb|\section{小節名稱}|"指令插入小節名稱,小節名稱會自動加到目錄中。
若目錄中想使用不同的小節名稱,則可以使用"\verb|\section[目錄顯示小節名稱]{內文顯示小節名稱}|"。


\section{如何插入圖片Insert figures}\label{sec:figure}

\section{標籤與引用 Label and references}\label{sec:Label_and_references}

\section{基本數學公式範例 Basic Example of Equation}\label{sec:Basic_Example_of_Equation}
